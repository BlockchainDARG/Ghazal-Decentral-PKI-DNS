% !TEX root = ../main.tex

\begin{abstract}

Recently, a number of projects (both from academia and industry) have examined decentralized public key infrastructures (PKI) based on blockchain technology. These projects vary in scope from full\-fledged domain name systems accompanied by a PKI to simpler transparency systems that augment the current HTTPS PKI. In this paper, we start by articulating, in a way we have not seen before, why this approach is more than a complementary composition of technologies, but actually a new and useful paradigm for thinking about who is actually authoritative over PKI information in the web certificate model. We then consider what smart contracts could add to the web certificate model, if we move beyond using a blockchain as passive, immutable (subject to consensus) store of data --- as is the approach taken by projects like Blockstack. To illustrate the potential, we develop and experiment with an Ethereum-based web certificate model we call \Ghazal, discuss different design decisions, and analyze deployment costs. 

\end{abstract}